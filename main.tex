% Latex template: mahmoud.s.fahmy@students.kasralainy.edu.eg
% For more details: https://www.sharelatex.com/learn/Beamer

% Slide Masters:

% Title
% Text
% 2 column
% Full-image
% Non-branded
% Bibliography
% Closing

\documentclass[aspectratio=1610]{beamer} % Aspect ratio
% https://tex.stackexchange.com/a/14339/5483 
% Possible values: 1610, 169, 149, 54, 43 and 32.
% 169 = 16:9

\usetheme{macquarie}

\usepackage[english]{babel}       % Set language
\usepackage[utf8x]{inputenc}      % Set encoding

\mode<presentation>           % Set options
{
  \usetheme{default}          % Set theme
  \usecolortheme{default}         % Set colors
  \usefonttheme{default}          % Set font theme
  \setbeamertemplate{caption}[numbered] % Set caption to be numbered
}

% Uncomment this to have the outline at the beginning of each section highlighted.
%\AtBeginSection[]
%{
%  \begin{frame}{Outline}
%    \tableofcontents[currentsection]
%  \end{frame}
%}

\usepackage{graphicx}         % For including figures
\usepackage{booktabs}         % For table rules
\usepackage{hyperref}         % For cross-referencing

\title{What is Git and Why is it Awesome?} % Presentation title
\author{Brian Ballsun-Stanton}               % Presentation author
\institute{Solutions Architect (Digital Humanities), Faculty of Arts}         % Author affiliation
\date{\today\\CC-BY\\https://github.com/Denubis/GitPresentationRUsers/main.pdf}                 % Today's date  

\begin{document}

% Title page
% This page includes the informations defined earlier including title, author/s, affiliation/s and the date
\begin{frame}
  \titlepage  
\end{frame}

% Outline
% This page includes the outline (Table of content) of the presentation. All sections and subsections will appear in the outline by default.
\begin{frame}{Outline}
  \tableofcontents
\end{frame}

% The following is the most frequently used slide types in beamer
% The slide structure is as follows:
%
%\begin{frame}{<slide-title>}
% <content>
%\end{frame}

\section{Concepts to untangle}

\begin{frame}{These things are different}
  
    \begin{itemize}
    \item Git
    \item Repository
    \item a DVCS
    \item A Remote Repository (Service)
    \item Extra Stuff
  \end{itemize}
\end{frame}


\begin{frame}{Git}
  
    \begin{itemize}
    \item Invented by Torvalds in 2005
    \item A response to prior version control systems that didn't scale
    \item Very counterintiutive terminology
    \item Looks scary    
  \end{itemize}
\end{frame}

\begin{frame}{Repository}
  
    \begin{itemize}
    \item A self-contained ``light-cone'' of code
    \item Able to talk to other repositories by exchanging ``commits''
    \item Every repository is master of its own fate
    \item Very different from a centralised version control system
    \item Complete opposite of Dropbox
  \end{itemize}
\end{frame}


\begin{frame}{Remote repositories}
  
    \begin{itemize}
    \item Github
    \item Gitlab
    \item Bitbucket

  \end{itemize}
\end{frame}

\begin{frame}{Extra Stuff}
  
    \begin{itemize}
    \item Git(hub/lab) Pages.io
    \item Large file store
    \item Continuous Integration


  \end{itemize}
\end{frame}


\begin{frame}{Why do we care?}
  
    \begin{itemize}
    \item Able to track changes,  
    \item On sets of files over time,
    \item With descriptive summaries!

  \end{itemize}
\end{frame}

\section{Scenarios for Awesomeness}


\begin{frame}{Thesis Work}
  
  \begin{itemize}
    \item https://github.com/Denubis/AskingAboutData
    \item https://github.com/petrajanouchova/dis\_context
    \begin{itemize}
        \item Reverting my bad changes:
      \item https://github.com/petrajanouchova/dis\_context/\\commit/3618ccc89149b6188fdeb4281c8da3a8df627ace
    \end{itemize}
  \end{itemize}
\end{frame}


\begin{frame}{Papers and Presentations}
  
    \begin{itemize}
    \item https://github.com/FAIMS/OpenDataSympathetic-Presentation/blob/master/presentation.pdf
    \item https://github.com/FAIMS/FAIMS-Mobile-Flexible-open-source-software-for-field-research
    \item Journal based uses:
    \begin{itemize}
      \item Open peer review: https://github.com/OpenScienceMOOC/Module-7-Open-Evaluation
      \item An entire journal: https://github.com/sx-archipelagos/sxa
    \end{itemize}
  \end{itemize}
\end{frame}

\begin{frame}{Code and Experiments}
  
  \begin{itemize}
    \item https://github.com/FAIMS
    \begin{itemize}  
      \item This code at this point in time: https://github.com/FAIMS/faims-web/releases/tag/ElsevierSOFTX-124 
      \item Supports this paper: https://www.sciencedirect.com/science/article/pii/S2352711017300869
    \end{itemize}    
    \item https://github.com/Alveo
  \end{itemize}
\end{frame}

\begin{frame}{Static Websites}
  
  \begin{itemize}
    \item https://github.com/orgs/BeniHassanResearchGroup/dashboard 
    \begin{itemize}
      \item -> www.benihassan.com
    \end{itemize}
    \item https://github.com/orgs/mqTeXUsers/dashboard 
    \begin{itemize}
      \item -> mqtexusers.github.io
    \end{itemize}
    \item https://github.com/orgs/MacquarieAustralianHistoryMuseum/dashboard 
    \begin{itemize}
      \item -> objectbasedlearning.com
    \end{itemize}
  \end{itemize}
\end{frame}

\section{Activities}


\begin{frame}{Signing up under an education link}
  
  \begin{itemize}
    \item education.github.com
    \item bitbucket.org with a .mq.edu.au email
    \item gitlab has all of these services free, we're working on .mq.edu.au support
  \end{itemize}
\end{frame}


\begin{frame}{SSH keyless authentication / GUI setup}
  
  \begin{itemize}
    \item https://help.github.com/articles/connecting-to-github-with-ssh/
    \item https://desktop.github.com
  \end{itemize}
\end{frame}

\begin{frame}{File sharing with code review}
  
  \begin{itemize}
    \item https://github.com/LibraryCarpentry/lc-webscraping/pull/27
    \item https://github.com/FAIMS/SoftwareX/commit/\\86355fedb0f7fca5fd6f65916882bf147ba73c63\#diff-9e1159339c97548e560d49d979f82b8e
    \item How would we get supervisors and gradstudents to use this model?
  \end{itemize}
\end{frame}

\begin{frame}{Getting a git pages site up}
  
  \begin{itemize}
    \item https://github.com/mqTeXUsers/mqTeXusers.github.io
    \item https://pages.github.com/
  \end{itemize}
\end{frame}

\begin{frame}{Osf.io Integration}
  The open science framework supports a github datasource.
  \begin{itemize}
    \item https://osf.io
    \item http://help.osf.io/m/addons/l/837075-connect-github-to-a-project
    \item example: https://osf.io/ncm8s/
  \end{itemize}
\end{frame}

\end{document}
