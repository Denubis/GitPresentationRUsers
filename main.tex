% Latex template: mahmoud.s.fahmy@students.kasralainy.edu.eg
% For more details: https://www.sharelatex.com/learn/Beamer

% Slide Masters:

%	Title
%	Text
%	2 column
%	Full-image
%	Non-branded
%	Bibliography
%	Closing

\documentclass[aspectratio=169]{beamer} % Aspect ratio
% https://tex.stackexchange.com/a/14339/5483 
% Possible values: 1610, 169, 149, 54, 43 and 32.
% 169 = 16:9

\usetheme{macquarie}

\usepackage[english]{babel}				% Set language
\usepackage[utf8x]{inputenc}			% Set encoding

\mode<presentation>						% Set options
{
  \usetheme{default}					% Set theme
  \usecolortheme{default} 				% Set colors
  \usefonttheme{default}  				% Set font theme
  \setbeamertemplate{caption}[numbered]	% Set caption to be numbered
}

% Uncomment this to have the outline at the beginning of each section highlighted.
%\AtBeginSection[]
%{
%  \begin{frame}{Outline}
%    \tableofcontents[currentsection]
%  \end{frame}
%}

\usepackage{graphicx}					% For including figures
\usepackage{booktabs}					% For table rules
\usepackage{hyperref}					% For cross-referencing

\title{Title for a minimal beamer presentation}	% Presentation title
\author{Author One}								% Presentation author
\institute{Name of institution}					% Author affiliation
\date{\today}									% Today's date	

\begin{document}

% Title page
% This page includes the informations defined earlier including title, author/s, affiliation/s and the date
\begin{frame}
  \titlepage

  
\end{frame}

% Outline
% This page includes the outline (Table of content) of the presentation. All sections and subsections will appear in the outline by default.
\begin{frame}{Outline}
  \tableofcontents
\end{frame}

% The following is the most frequently used slide types in beamer
% The slide structure is as follows:
%
%\begin{frame}{<slide-title>}
%	<content>
%\end{frame}

\section{Section One}

\begin{frame}{Slide with bullet points}
	This is a bullet list of two points:
    \begin{itemize}
		\item Point one
        \item Point two
	\end{itemize}
\end{frame}

\begin{frame}{Slide with two columns}
	\begin{columns}
		\column{.5\textwidth}
        Text goes in first column.
        
        \column{.5\textwidth}
        Text goes in second column
	\end{columns}
\end{frame}

\section{Section Two}

\begin{frame}{Slide with table}
	% Please add the following required packages to your document preamble:
% \usepackage{booktabs}
\begin{table}[H]
\centering
\caption{Caption for table one}
\label{tab:table1}
\begin{tabular}{@{}lcc@{}}
\toprule
\textbf{Heading1} & \textbf{Heading2} & \textbf{Heading3} \\ \midrule
Row1 & 0.1 & .01 \\
Row2 & 0.2 & .02 \\
Row3 & 0.3 & 0.03 \\
Row4 & 0.4 & 0.04 \\ \bottomrule
\end{tabular}
\end{table}
\end{frame}

\begin{frame}{Slide with figure}
	\begin{figure}[H]
		\centering
        \includegraphics[width=.5\textwidth]{demo/logo.eps}
        \caption{Caption for figure one.}
        \label{fig:figure1}
	\end{figure}
\end{frame}

\begin{frame}{Slide with references}
	This is to reference a figure (Figure \ref{fig:figure1})\\
    This it to reference a table (Table \ref{tab:table1})\\
    This is to cite an article \cite{Ahmed2018a}\\
    This is to add an article to the references without mentioning in the text \nocite{Ahmed2018a}\\
\end{frame}
\section{References}

% Adding the option 'allowframebreaks' allows the contents of the slide to be expanded in more than one slide.
\begin{frame}[allowframebreaks]{References}
	\tiny\bibliography{references}
	\bibliographystyle{apalike}
\end{frame}

\end{document}
